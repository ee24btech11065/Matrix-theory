%iffalse
\let\negmedspace\undefined
\let\negthickspace\undefined
\documentclass[journal,12pt,onecolumn]{IEEEtran}
\usepackage{cite}
\usepackage{amsmath,amssymb,amsfonts,amsthm}
\usepackage{algorithmic}
\usepackage{graphicx}
\usepackage{textcomp}
\usepackage{xcolor}
\usepackage{txfonts}
\usepackage{listings}
\usepackage{enumitem}
\usepackage{mathtools}
\usepackage{gensymb}
\usepackage{comment}
\usepackage[breaklinks=true]{hyperref}
\usepackage{tkz-euclide} 
\usepackage{listings}
\usepackage[latin1]{inputenc}
\usepackage{gvv}                                        
%\def\inputGnumericTable{}                                 
                               
\usepackage{color}                                            
\usepackage{array}                                            
\usepackage{longtable}                                       
\usepackage{calc}                                             
\usepackage{multirow}                                         
\usepackage{hhline}                                           
\usepackage{ifthen}                                           
\usepackage{lscape}
\usepackage{tabularx}
\usepackage{array}
\usepackage{float}
\usepackage[utf8]{inputenc}
\usepackage{multicol}

\newtheorem{theorem}{Theorem}[section]
\newtheorem{problem}{Problem}
\newtheorem{proposition}{Proposition}[section]
\newtheorem{lemma}{Lemma}[section]
\newtheorem{corollary}[theorem]{Corollary}
\newtheorem{example}{Example}[section]
\newtheorem{definition}[problem]{Definition}
\newcommand{\BEQA}{\begin{eqnarray}}
\newcommand{\EEQA}{\end{eqnarray}}
\newcommand{\define}{\stackrel{\triangle}{=}}
\theoremstyle{remark}
\newtheorem{rem}{Remark}

% Marks the beginning of the document

\begin{document}
\bibliographystyle{IEEEtran}


\title{16. Applications of derivatives}
\author{EE24BTECH11065 - spoorthi }
\newpage
\maketitle


\bigskip

\renewcommand{\thefigure}{\theenumi}
\renewcommand{\thetable}{\theenumi}
\begin{large}
\textbf{{Section-B JEE Main/AIEEE}}
\end{large}


\begin{enumerate}



    \item A spherical iron ball 10$cm$ in radius is coated with a layer of ice of uniform thickness that melts at a rate of 50 $cm^3$ /min. When the thickness of ice is 5$cm$,then the rate at which the thickness of ice decreases is\hfill\sbrak{2005}
    \begin{multicols}{4}
    \begin{enumerate}
        \item 
        $\frac{1}{36\pi}$ $cm/min$. 
        \item $\frac{1}{18\pi}$ $cm/min$. 
        \item $\frac{1}{54\pi}$ $cm/min$. 
        \item $\frac{5}{6\pi}$  $cm/min$. 
        
    \end{enumerate}
    \end{multicols}

    
    \item If the equation $a_nx^n$ + $a$$_{n-1}x^{n-1}$ + .......... + $a_1$$x$ $=$ 0, $a_1$ $\ne$ 0,$n$$\geq$2, has a positive root $x$ = $\alpha$, then the equation $n$$a_n$$x^{n-1}$  $+$ $\brak{n_1}$$a_{n-1}$$x^{n-2}$ +....... + $a_1$ $=$0 has a positive root, which is 
    \hfill\sbrak{2005}
     
    \begin{enumerate}
    
        
    
    
        \item greater than $\alpha$
        \item smaller than $\alpha$
        \item greater than or equal to $\alpha$
        \item equal to $\alpha$
        
        \end{enumerate}
        
            
        \item The function $f(x)$ = $\frac{x}{2}$ + $\frac{2}{x}$ has a local minimum at \hfill\sbrak{2006}
         \begin{multicols}{4}
        \begin{enumerate}
        
            
        
         \item $x$ = 2
        \item $x$ = -2
        \item $x$ = 0
        \item $x$ = 1
        
            
        \end{enumerate}
        \end{multicols}
            
        \item A triangular park is enclosed on two sides by a fence and on third side by a straight river bank. The two sides having fence are of same length $x$.  The maximum area enclosed by the park is \hfill\sbrak{2006}
         \begin{multicols}{4}
        \begin{enumerate}
        
            
        
            \item $\frac{3}{2}$$x^2$
        \item $\sqrt{\frac{x^3}{8}}$
        \item $\frac{1}{2}$$x^2$
        \item $\pi$$x^2$
    
        \end{enumerate}
        \end{multicols}
            
        \item A value of $c$ for which conclusion of Mean Value Theorem holds for the function $f(x)$ = $\log_e$ $x$ on the interval $\sbrak{1,3}$ is \hfill\sbrak{2007}
         \begin{multicols}{4}
        \begin{enumerate}
        
            
        
            \item $\log_3$$e$
        \item $\log_e$3
        \item 2$\log_3$$e$
        \item $\frac{1}{2}$$\log_3$$e$

        \end{enumerate}
        \end{multicols}
            
        \item The function $f(x)$ = $\tan^{-1}\brak{\sin{x} + \cos{x}}$is an increasing function in \hfill\sbrak{2007}
         \begin{multicols}{4}
    
        \begin{enumerate}
        
            
        
      \item $\brak{0,\frac{\pi}{2}}$
        \item $\brak{\frac{-\pi}{2},\frac{\pi}{2}}$
        \item $\brak{\frac{\pi}{4},\frac{\pi}{2}}$
        \item $\brak{\frac{-\pi}{2},\frac{\pi}{4}}$
        
            
        
            
        
        \end{enumerate}
        \end{multicols}
            
        \item If $p$ and $q$ are positive real numbers such that $p^2$ + $q^2$ = 1, then the maximum value of $\brak{p + q}$ is\hfill\sbrak{2007}
         \begin{multicols}{4}
        \begin{enumerate}
        
            

        \item $\frac{1}{2}$
        \item $\frac{1}{\sqrt{2}}$
        \item $\sqrt{2}$
        \item 2
        \end{enumerate}
        \end{multicols}
            
        \item Suppose the cubic $x^3$ - $p$$x$ + $q$ has three distinct real roots where $p$ $>$ 0 and $q$ $>$ 0. Then which one of the following holds ? \hfill\sbrak{2008}
         
        \begin{enumerate}
        
            
        
        \item The cubic has minima at $\sqrt{\frac{p}{3}}$ and maxima at -$\sqrt{\frac{p}{3}}$
        \item The cubic has minima at -$\sqrt{\frac{p}{3}}$ and maxima at $\sqrt{\frac{p}{3}}$
        \item The cubic has minima at both $\sqrt{\frac{p}{3}}$ and -$\sqrt{\frac{p}{3}}$
        \item the cubic has maxima at both $\sqrt{\frac{p}{3}}$ and -$\sqrt{\frac{p}{3}}$
        
        
        \end{enumerate}
        
            
        \item How many real solutions does the equation $ x^7 + 14x^5 + 16x^3 + 30x- 560 = 0$ have ? \hfill\sbrak{2008}
         \begin{multicols}{4}
        \begin{enumerate}
    
            
        
            \item 7
        \item 1
        \item 3
        \item 5
        
        \end{enumerate}
        \end{multicols}
            
        \item Let $f\brak{x}$= $x$$\abs{x}$ and $g\brak{x}$ = $\sin{x}$. 
        
    \textbf{statement-1:}  $gof$ is differentiable at $x=0$ and its derivative is continuous at that point. 
            \textbf{statement-2:  }$gof$ is twice differential at $x = 0$.  
                 \hfill\sbrak{2009} 
                  
        
    
        \begin{enumerate}
        
            
        
            \item Statement-1 is true, Statement-2 is true; statement-2 is not a correct explanation for statement-1.  
            
        \item Statement-1 is true, Statement-2 is false.  
        \item Statement-1 is false,Statement-2 is true.  
        \item Statement-1 is true, Statement-2 is true;Statement-2 is a correct explanation for Statement-1.  

        \end{enumerate}
        
            
        \item Given $ P\brak{x}=x^4+ ax^3+bx^2+ cx+d$ such that $x = 0$ is the only real root of $P^\prime\brak{x} = 0$. If $P\brak{-1}$ $<$ $P\brak{1}$, then in the interval $\sbrak{-1,1}$ $\colon$ \hfill\sbrak{2009}
        
        \begin{enumerate}
        
            
        
            
        \item $P\brak{-1}$ is not minimum but $P\brak{1}$ is the maximum of $P$. 
        \item $P\brak{-1}$ is the minimum but $P\brak{1}$ is not the maximum of $P$. 
        \item  Neither $P\brak{-1}$ is the  minimum nor  $P\brak{1}$ is the maximum of $P$. 
        \item $P\brak{-1}$ is the minimum and $P\brak{1}$ is the maximum of $P$. 
        
        \end{enumerate}
        
            
        \item The equation of the tangent to the curve $y$ = $x$ + $\frac{4}{x^2}$ , that is parallel to the $x$-axis, is\hfill\sbrak{2010}
         \begin{multicols}{4}
        \begin{enumerate}
        
            
        
            \item $y=1$
        \item $y=2$
        \item $y=3$
        \item $y=0$
        \end{enumerate}
        \end{multicols}
            
        \item Let $f$$\colon$ $R$ $\to$ $R$ be defined by $f\brak{x}$ = 
    
 $\left\{ \begin{array}{rcl}
     k-2x & \text{if} &  x\leq -1 \\ 2x+3 & \text{if} & x>-1
     \end{array}\right.$
        
        
        If $f$ has a local minimum at $x = -1$,then a possible value of $k$ is\hfill\sbrak{2010}
         \begin{multicols}{4}
        
        
        \begin{enumerate}
        
            
        
        \item 0
        \item -$\frac{1}{2}$
        \item -1
        \item 1
        
        \end{enumerate}
        \end{multicols}
            
     \item Let $f$$\colon$ $R$ $\to$ $R$ be a continuous function defined by $f\brak{x}$ = $\frac{1}{e^x + 2e^{-x}}$ \hfill\sbrak{2010}
      
        \begin{enumerate}
    
            
        
            
            \textbf{Statement-1} $\colon$ $f(c)$ = $\frac{1}{3}$ , for some $c$$\in$$R$.
            \textbf{Statement-2}$\colon$ 0$<$$f(x)$$\leq$$\frac{1}{2\sqrt{2}}$, for all $x$$\in$ $R$
            
       
        \end{enumerate}
         
            \begin{enumerate}
            
                
            
                \item Statement-1 is true, Statement-2 is true;Statement-2 is \textbf{not} a correct explanation for statement-1. 
        \item Statement-1 is true, Statement-2 is false.  
        \item Statement-1 is false, Statement-2 is true.  
        \item Statement-1 is true,Statement 2 is true; Statement-2 is a correct explanation for statement-1. 
    
        
            \end{enumerate}
           
            
            \item The shortest distance between line $y-x=1$ and curve $x$=$y^2$ is \hfill\sbrak{2011}
             \begin{multicols}{4}
            \begin{enumerate}
            
                
    \item $\frac{3\sqrt{2}}{8}$
        \item $\frac{8}{3\sqrt{2}}$
        \item $\frac{4}{\sqrt{3}}$
        \item $\frac{\sqrt{3}}{4}$
        
            \end{enumerate}
            \end{multicols}
            
    
        
        
    
    \end{enumerate}


\end{document}
